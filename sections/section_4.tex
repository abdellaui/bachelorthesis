% !TEX root = ../my_thesis.tex

\section{Evaluation}
Im vorliegenden Kapitel werden die Ergebnisse der durchgeführten Experimente präsentiert. Zuallererst wurden die Ergebnisse von BIM-PoseNet \cite{acharyaBIMPoseNetIndoorCamera2019} reproduziert, um die Korrektheit der Pipeline zu überprüfen. Daher werden die Ergebnisse der Reproduktion zuerst präsentiert. Anschließend wird der Hyperparameter $\beta$ der Kostenfunktion (vgl. Gleichung \ref{eq:posenet_loss}) für jedes Datensatz (siehe Abschnitt \ref{subsec:datasets}) bestimmt und angegeben. Daraufhin werden die Evaluationsergebnisse der jeweiligen Datensätze dargestellt.
\subsection{Reproduktion der Ergebnisse von BIM-PoseNet}
Die Ergebnissen der Experimente von \citet{acharyaBIMPoseNetIndoorCamera2019}, die das PoseNet Model mit Gradientenbilder der karikaturistischen Daten sowie synthetischen Kantenbilder trainieren und anschließend mit den Gradientenbilder der realen Daten evaluieren, konnten näherungsweise (vgl. Tabelle \ref{tab:reproduction}) reproduziert werden. Eine exakte oder bessere Reproduktion der Ergebnisse ist durch Zufall bedingt und wird in dieser Arbeit aus Zeitgründen vernachlässigt.

Abweichend von BIM-PoseNet wurden statt 1000 reale Bilder 600 reale Bilder evaluiert, weil derzeit 600 Evaluierungsbilder veröffentlicht sind. Die Mengenunterschiede der Evaluierungsdaten sind für die Endergebnisse trivial, da die Akkuratesse sich aus den Durchschnittswerten den Evaluationsergebnissen zusammensetzt und die Daten in zufälliger Reihe evaluiert werden. 

Der Trainingsprozess wurde je Datensatztyp zweimal wiederholt und die besseren Evalutaionsergebnisse behalten. Die Tabelle \ref{tab:reproduction} präsentiert die Ergebnisse der Reproduktion sowie die Ergebnisse der Autoren \citet{acharyaBIMPoseNetIndoorCamera2019}.


\begin{table}[H]
	\centering
	\caption{Reproduktionsergebnisse. Abweichungen der Ergebnisse sind durch Zufall bedingt und können bei mehrfachem Wiederholen des Trainingsprozesses minimiert bzw. erhoben sowie verbessert werden. }
	\begin{tabularx}{1.0\textwidth}{>{\hsize=0.7\hsize}X >{\hsize=1.3\hsize}X X}
		\textbf{Quelle} & \textbf{Trainingsdatensatz} \hspace{2cm} (Gradientenbild)& \textbf{Akkuratesse} \hspace{2cm} (Position, Orientierung)\\
		\hline
		BIM-PoseNet & karikaturistische Simulation & 2.63$m$, 6.99°\\
		\hline
		BIM-PoseNet & synthetisches Kantenbild & 1.88$m$, 7.73°\\
		\hline
		Reproduktion & karikaturistische Simulation & 2.74$m$, 12.24°\\
		\hline
		Reproduktion & synthetisches Kantenbild & 2.53$m$, 9.54°\\
	\end{tabularx}
	\label{tab:reproduction}
\end{table}



\subsection{Bestimmung des Hyperparameters $\beta$}
\label{subsec:determine_beta}
gridsearch je datensatz mit realdaten, empfohlen 120-750 in gebäuden, in 9 schritten je 70 stride; plote ergebnisse matplotlib
\subsubsection{IC-loop}
\subsubsection{HS-stairs-up}
\subsubsection{HS-stairs-down}
\subsubsection{HS-gamma}

\subsection{Ergebnisse: IC-Flur }
badabum
\subsection{Ergebnisse: Seminargebäude Stairs}
badabäm
\subsection{Ergebnisse: Seminargebäude Floor}
badabum
