% !TEX root = ../my_thesis.tex
\pagebreak
\section{Stand der Forschung und Grundlagen}

Dieses Kapitel versucht einen Überblick des Forschungsstandes in den unterschiedlichen Aspekten der Arbeit zu verschaffen. Anschließend vermittelt das Kapitel notwendige Grundkenntnisse.

\subsection{Einführung}
% Lokalisierungsproblem erläutern
% Ansätze aufzählen
% Überführen auf visuelle Lokalisierung

% Arten erläutern
% indirekte methoden
% direkte methoden
% - point-matching
% - rgb-d
% - image only
Visuelle Lokalisierung, engl. Visual-Based Localization(VBL), wird als die Bestimmung der Pose (Position + Orientierung) eines visuellen Abfragemateriales (z.B. Bildes) in einer bekannten Szene verstanden \cite{piascoSurveyVisualBasedLocalization2018}.
\textbf{Pose Estimation} wird in der
bla bla bla



Visual-Based Localization (VBL) consists of retrieving the pose (position + orientation) of a visual query
material within a known space representation.  \cite{piascoSurveyVisualBasedLocalization2018}

When designing a VBL system, the type of method is not the only parameter to consider. As pointed
out in [111], robustness to environment appearance changes over time is a main concern


Visual Place Recognition is a roboticist problem, defined in the general sense in [111] as the visual ability
of a human, an animal or a robot to recognize an already visited place



% feature matching
% Features to Points matching

VBL 



\\\\
% Überführen auf CNNs
\textbf{Convolutional Neural Networks} \textit{(CNN)} werden erfolgreich im Bereich des Maschinelles Sehens, wie z. B. bei der Klassifizierung von Bildern \cite{krizhevskyImageNetClassificationDeep2012, simonyanVeryDeepConvolutional2014, heDeepResidualLearning2015} sowie bei der Objekterkennung \cite{girshickRichFeatureHierarchies2013, renFasterRCNNRealTime2015b, girshickFastRCNN2015} eingesetzt. 
Ein verbreiteter Ansatz beim Entwurf von CNNs ist das häufig zweckentfremdende Feintunen \textit{(fine-tune)} der Netzwerkarchitekturen, die z. B. für die Bildklassifizierung angesichts der Aufgaben von ImageNet \cite{russakovskyImageNetLargeScale2014} konstruiert wurden. Dieser Ansatz konnte beispielsweise erfolgreich in der Objekterkennung \cite{girshickFastRCNN2015}, Objektsegmentierung \cite{kokkinosPushingBoundariesBoundary2015, maninisConvolutionalOrientedBoundaries2016}, semantische Segmentierung \cite{nohLearningDeconvolutionNetwork2015, hazirbasFuseNetIncorporatingDepth2017a} und Tiefenbestimmung \cite{liDepthSurfaceNormal2015} verfolgt werden.
Seit Kurzem werden CNNs auch in den Anwendungsgebieten der Lokalisierung verwendet. Zum Beispiel verwenden Parisotto \etal\cite{parisottoGlobalPoseEstimation2018} CNNs in Bezug auf das SLAM Problem. Melekhov \etal\cite{melekhovRelativeCameraPose2017} schätzen anhand CNNs die relative Pose zweier Kameras. Constante \etal\cite{costanteExploringRepresentationLearning2016} und Wang \etal\cite{wangDeepVOEndtoendVisual2017} setzen es im Bereich der visuellen Odometrie ein.

% Posenet überführen & erklären
Geleitet von den \textit{state-of-the-art} Lokalisierungsergebnissen der CNNs stellen Kendall \etal\cite{kendallPoseNetConvolutionalNetwork2015} den ersten Ansatz zu direkten Posebestimmung vor. PoseNet \cite{kendallPoseNetConvolutionalNetwork2015} ist die Modifikation der GoogLeNet \cite{szegedyGoingDeeperConvolutions2015} Architektur und zweckentfremdet es von der Bildklassifizierung zu einem Pose-Regressor. Trainiert mit einem Datensatz, bestehend aus Paaren von Farbbild und Pose, kann es die sechs Freiheitsgrade der Kameraposen in unbekannten Szenen mittels eines Bildes bestimmen. Dieser Ansatz benötigt weder Tiefenbilder der Szene noch eine durchsuchbare Bildgalerie. Im Vergleich zu den metrischen Ansätzen wie SLAM oder visuelle Odometrie liefert es eine weniger akkurate Pose. Es bietet jedoch eine hohe Toleranz gegenüber Skalierungs- und Erscheinungsänderungen des Anfragebildes an \cite{piascoSurveyVisualBasedLocalization2018}.

% Varianten aufzählen
Es gibt mehrere Ansätze, die die Genauigkeit von PoseNet  \cite{kendallPoseNetConvolutionalNetwork2015} übertreffen.
Einen Fortschritt erhalten die Autoren von PoseNet \cite{kendallPoseNetConvolutionalNetwork2015} durch die hier \cite{kendallModellingUncertaintyDeep2015a} vorgestellte Anpassung ihres Models an einem Bayessian Neural Network \cite{denkerTransformingNeuralNetOutput1991, mackayPracticalBayesianFramework1991}.
Dieselben Autoren erweitern PoseNet \cite{kendallPoseNetConvolutionalNetwork2015} mit einer neuen Kostenfunktion unter Berücksichtigung von geometrischen Eigenschaften \cite{kendallGeometricLossFunctions2017}. Wlach \etal\cite{walchImagebasedLocalizationUsing2016} und Clark \etal\cite{clarkVidLocDeepSpatioTemporal2017} setzen Long-Short-Term-Memory \textit{(LSTM)} \cite{hochreiterLongShortTermMemory1997a} Einheiten ein, um Wissen aus der Korrelation von Bildsequenzen zu gewinnen. Wu \etal\cite{wuDelvingDeeperConvolutional2017} und Naseer \etal\cite{naseerDeepRegressionMonocular2017} augmentieren den Trainingsdatensatz. Wu \etal\cite{wuDelvingDeeperConvolutional2017} stocken den vorhandenen Datensatz auf, indem sie die Bilder künstlich rotieren. Naseer \etal\cite{naseerDeepRegressionMonocular2017} erweitern zuerst über ein weiteres CNN den Datensatz um Tiefenbildern. Anschließend simulieren sie RGB-Bilder aus verschiedenen Viewpoints. Im Vergleich zu PoseNet  \cite{kendallPoseNetConvolutionalNetwork2015} verwenden Müller \etal\cite{mullerSQUEEZEPOSENETIMAGEBASED2017} und Melekhov \etal\cite{melekhovImageBasedLocalizationUsing2017} eine andere Architektur. 
Das Modell von Müller \etal\cite{mullerSQUEEZEPOSENETIMAGEBASED2017} basiert auf die SqueezeNet \cite{iandolaSqueezeNetAlexNetlevelAccuracy2016} Architektur. Melekhov \etal stellen HourglassNet \cite{melekhovImageBasedLocalizationUsing2017}, basierend auf einem symmetrischen Encoder-Decoder Architektur, vor. Brahmbhatt \etal\cite{brahmbhattGeometryAwareLearningMaps2018} und Valada \etal\cite{valadaDeepAuxiliaryLearning2018, valadaIncorporatingSemanticGeometric} binden zusätzliche Informationen wie z.B. visuelle Odometrie, GPS oder IMU ein. 

Jedes dieser Ansätze benötigen annotierte Traininsdaten. Für die Erstellung solcher Daten wurden beispielsweise mit entsprechender Hardware ausgerüstete Trolleys \cite{huitlTUMindoorExtensiveImage2012}, 3D-Kameras \cite{izadiKinectFusionRealtime3D2011} oder SfM-Methoden \cite{kendallPoseNetConvolutionalNetwork2015} eingesetzt.
\\\\
% Simulierte 3D-Daten 
\textbf{Simulierte 3D-Daten} werden in der Literatur oft eingesetzt, um das manuelle Erzeugen und Annotieren von Daten umzugehen. Pishchulin \etal\cite{pishchulinArticulatedPeopleDetection2012a}, Peng \etal\cite{pengLearningDeepObject2014}, Su \etal\cite{suRenderCNNViewpoint2015} und Varol \etal\cite{varolLearningSyntheticHumans2017} erzeugen ihren Trainingsdaten, indem sie virtuelle Objekte auf reale Hintergrundbildern platzieren. Pishchulin \etal\cite{pishchulinArticulatedPeopleDetection2012a} generieren Daten zwecks Personenerkennung und Bestimmung derer körperlicher Pose. Zuvor werden auf den vorhandenen Bildern die körperliche Pose der Personen bestimmt und daran deren 3D Modelle rekonstruiert. Anschließend werden die 3D-Modelle in ihrer Pose variiert auf reale Hintergrundbildern platziert. Peng \etal\cite{pengLearningDeepObject2014} erstellen Daten um Objekte auf realen Bildern zu detektieren. Von jeder Objektklasse werden 3D-Modelle auf einem Hintergrundbild aus einer Sammlung gelegt. Su \etal\cite{suRenderCNNViewpoint2015} generieren einen großen Datensatz mit 3D-Modellen, um den Viewpoint von Objekten auf realen Bildern zu bestimmen. Bei dieser Datengenerierung wird jedes virtuelle Objekt auf zufällige Hintergrundbildern positioniert und mit unterschiedlichen Konfigurationen \textit{(wie z.B. Beleuchtung)} gerendert. 
Varol \etal\cite{varolLearningSyntheticHumans2017} erstellen künstliche Personen auf Bildern, um beispielsweise den menschlichen Körper in seine Glieder zu segmentieren. Dabei rendern sie zufällige virtuelle Personen mit zufälliger Pose auf beliebige Hintergrundbildern.
Fanello \etal\cite{fanelloLearningBeDepth2014} rendert künstliche Infrarotbilder von Händen und Gesichtern zwecks Tiefenerkennung und Segmentierung der Hand in den einzelnen Fingern sowie des Gesichtes in Bereiche aus einem RGB-Bild.
Dosovitskiy \etal\cite{dosovitskiyFlowNetLearningOptical2015} erlernen mit synthetischen Daten den optischen Fluss von Bildsequenzen.  Hierbei werden auf Hintergrundbildern aus einer Sammlung mehrmals bewegte virtuelle Stühle platziert.

Motiviert von der Datengenerierung über 3D-simulierten Daten stellt Ha \etal\cite{haImagebasedIndoorLocalization2018} einen Ansatz zur Bild-basierte Lokalisierung in Gebäuden vor. Dieser Forschungsansatz generiert synthetische Daten aus einem Building Information Modeling \textit{(BIM)}. Bei den Daten werden die durch das vortrainierte VGG Netzwerk \cite{simonyanVeryDeepConvolutional2014} extrahierte Features als wesentlich erachtet und in einer Datenbank gepflegt. Ein reales Aufnahmebild im Gebäude lässt sich durch den Vergleich der Features lokalisieren. Acharya \etal\cite{acharyaBIMPoseNetIndoorCamera2019, acharyaMODELLINGUNCERTAINTYSINGLE2019} erzeugen ebenso Trainingsdaten aus einem BIM, jedoch verwenden sie zur Lokalisierung keine Datenbank bedürftiges Verfahren, sondern bestimmen die Pose direkt über PoseNet \cite{kendallPoseNetConvolutionalNetwork2015}. Die Daten werden entlang eines Flugbahnes aus der Simulation eines Korridors gesammelt. Hierbei werden sich in der Realitätstreue vom karikaturistisch zu fotorealistisch hin über zu fotorealistisch-texturiert unterscheidende Daten erzeugt. Die besten Ergebnisse konnten die Autoren trainiert mit den Gradienten- und Kantenbilder der karikaturistischen Daten, getestet auf die Gradientenbilder der realen Aufnahmen, erzielen. Bei der Erzeugung von Gradienten- bzw. Kantenbilder gehen einerseits wichtige Informationen im Hinblick auf das Ursprungsbild verloren andererseits bleiben wichtige Informationen wie z. B. die geometrische Struktur erhalten.
\newline
\newline
Im weiteren Fortgang des Kapitels werden einige grundlegende Themen erläutert. Zuerst wird die Lineare Faltung erklärt und die Verarbeitung eines Bildes über den Sobelfilter zum Gradientenbild ausgeführt. Danach wird ein vertieftes Wissen an CNN vermittelt und anschließend bekannte CNN Modelle näher erläutert.

\subsection{Lineare Faltung}
\subsubsection{Sobelfilter}
\subsubsection{Gradientenbild}
\subsection{Convulutional Neural Network}
\subsubsection{Convolution Layer}
\subsubsection{Pooling Layer}
\subsubsection{Fully Connected Layer}
\subsection{Bekannte CNN Modelle}
\subsubsection{GoogLeNet}
\subsubsection{PoseNet}

\pagebreak