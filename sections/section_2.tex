% !TEX root = ../my_thesis.tex
\pagebreak
\section{Grundlagen}



% Lokalisierungsproblem erläutern
% Ansätze aufzählen
% Überführen auf visuelle Lokalisierung

% Arten erläutern

% Überführen auf CNNs
\pagebreak
In der Objekterkennung \cite{girshickRichFeatureHierarchies2013, girshickFastRCNN2015} sowie bei der Klassifizierung von Bildern \cite{krizhevskyImageNetClassificationDeep2012, heDeepResidualLearning2015} werden CNNs erfolgreich eingesetzt.
Seit Kurzem werden CNNs auch im Anwendungsgebieten der Lokalisierung verwendet. Zum Beispiel verwenden \textit{Parisotto et al.} CNNs in Bezug auf das SLAM Problem \cite{parisottoGlobalPoseEstimation2018}, \textit{Melekhov et al.} schätzen anhand CNNs die relative Pose zweier Kameras \cite{melekhovRelativeCameraPose2017}. \textit{Constante et al.} und \textit{Wang et el.} setzen es im Bereich der visuellen Odometrie ein \cite{costanteExploringRepresentationLearning2016, wangDeepVOEndtoendVisual2017}.

% Posenet überführen
Geleitet von state-of-the-art Ergebnissen der CNNs stellen \textit{Kendall et al.} PoseNet vor.
% Posenet erklären

% Varianten aufzählen
Für die Vebesserung der Lokalisierung von PoseNet wurden viele Ansätze vorgestellt. Beispiel hierfür sind.



% Alternative CNNs regressors
Alternative 
% Simulierte Daten / syntethic

% Machine learning + syntethic

% 
