% !TEX root = ../my_thesis.tex
\pagebreak
\section{Grundlagen}



% Lokalisierungsproblem erläutern
% Ansätze aufzählen
% Überführen auf visuelle Lokalisierung

% Arten erläutern
% indirekte methoden
% direkte methoden
% - point-matching
% - rgb-d
% - image only
% Überführen auf CNNs
\pagebreak
DCNNs werden erfolgreich im Bereich des Maschinelles Sehen, wie z.B. bei der Klassifizierung von Bildern \cite{krizhevskyImageNetClassificationDeep2012, simonyanVeryDeepConvolutional2014, heDeepResidualLearning2015} sowie bei der  Objekterkennung \cite{girshickRichFeatureHierarchies2013, renFasterRCNNRealTime2015b, girshickFastRCNN2015},  eingesetzt. 

Ein verbreiteter Ansatz beim Entwurf von DCNNs ist das Feinabstimmen (fine-tune) der Netzwerkarchitekturen, die für verschiedene Aufgaben der Bildklassifizierung von ImageNet \cite{russakovskyImageNetLargeScale2014} konstruiert wurden. Dieser Ansatz konnte beispielsweise erfolgreich in der Objekterkennung \cite{girshickFastRCNN2015}, Objektsegmentierung \cite{kokkinosPushingBoundariesBoundary2015, maninisConvolutionalOrientedBoundaries2016}, Semantische Segmentierung \cite{nohLearningDeconvolutionNetwork2015, hazirbasFuseNetIncorporatingDepth2017a} und Tiefenbestimmung \cite{liDepthSurfaceNormal2015} verfolgt werden.

Seit Kurzem werden DCNNs auch in den Anwendungsgebieten der Lokalisierung verwendet. Zum Beispiel verwenden \textit{Parisotto et al.} DCNNs in Bezug auf das SLAM Problem \cite{parisottoGlobalPoseEstimation2018}. \textit{Melekhov et al.} schätzen anhand DCNNs die relative Pose zweier Kameras \cite{melekwashovRelativeCameraPose2017}. \textit{Constante et al.} und \textit{Wang et el.} setzen es im Bereich der visuellen Odometrie ein \cite{costanteExploringRepresentationLearning2016, wangDeepVOEndtoendVisual2017}.
\\\\
% Posenet überführen
Geleitet von state-of-the-art Ergebnissen der CNNs stellen \textit{Kendall et al.} PoseNet \cite{kendallPoseNetConvolutionalNetwork2015} vor.
% Posenet erklären
PoseNet 
Das Netzwerk wird mit Bild und Posenpaare direkt trainiert.

% Varianten aufzählen
Für die Verbesserung der Lokalisierung von PoseNet wurden viele Ansätze vorgestellt. Eine Erweiterung von PoseNet stellen die Autoren selbst


% Alternative CNNs regressors
Alternative 

% Simulierte Daten / syntethic

% Machine learning + syntethic

% 
