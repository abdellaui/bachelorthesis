% !TEX root = ../my_thesis.tex

\section{Diskussion}
\label{sec:kapitel_5}



%Angesichts dieser Ergebnisse liegt die Schlussfolgerung nahe, dass das Evaluieren mit den Gradientenbilder der realen Daten auf das mit synthetischen Daten trainierten PoseNet.
%Es ist nicht das Ziel dieser Arbeit den visuellen Lokalisierungsansatz von \citet{acharyaBIMPoseNetIndoorCamera2019} auf den erhobenen Datensätzen zu untersuchen. Dabei wird eine Lokalisierungsgenauigkeit im Meterbereich angestrebt, da PLACEHOLDER . Daher ist die Abweichung (\textit{Drift}) bis zu 5\% der realen Evaluationsdaten von den synthetischen Trainingsdaten unproblematisch. In Anbetracht der Ergebnisse ist eine Lokalisierungsverfahren ohne weitere Maßnahmen des Ansatzes von \citet{acharyaBIMPoseNetIndoorCamera2019} auf den in dieser Arbeit erhobenen Datensätzen auszuschließen.


% Zusammenfassung Ergebnisse
Die Ergebnisse der Untersuchungen haben gezeigt, dass der Ansatz von \citet{acharyaBIMPoseNetIndoorCamera2019}, worin das PoseNet mit Gradietenbilder der synthetischen Daten trainiert und mit Gradientenbildern der realen Daten evaluiert wird, ohne Optimierung der Hyperparameter auf die im Rahmen dieser Bachelorarbeit erhobenen Datensätze eine durchschnittliche Akkuratesse von 14.61$m$ in der Position und 66.26° in der Orientierung erreichte. Die Evaluationen mit den synthetischen Gradientenbildern haben gezeigt, dass mit Daten aus der gleichen Domäne eine durchschnittliche Akkuratesse von 1.59$m$, 11.14° erzielt werden kann. 


% Perceptual aliasing & treppen
In Gebäuden kommt es häufiger vor, dass unterschiedliche Stellen eines Innenraumes ähnliche Merkmale besitzen und somit schwer voneinander zu unterscheiden sind. Dieses Problem ist in der Literatur auch als \textit{perceptual-aliasing} bekannt und stellt einen der größten Herausforderungen von Lokalisierungsverfahren dar \cite{lowryVisualPlaceRecognition2016}. Auf den \textit{HS-stairs-down} und \textit{HS-stairs-up} Strecken wurden die realen Evaluationsdaten grundsätzlich zwischen der oberen und unteren Treppenlauf lokalisiert. Eine Generalisierungsfähigkeit zwischen Oben und Unten der KNNs ist nicht zuerkennen. Vielmehr ist eine \textit{zufällige} Zuordnung der KNNs festzuhalten. Dies kann auf perceptual-aliasing zurückgeführt werden und erklärt die abwechselnde Fehlerrate auf den Evaluationsdaten der Treppenläufe.

% loop und gamma auf muster aufmerksam machen  & STRECKE VON ACHARYA
Weiterhin haben die Ergebnisse der Strecken \textit{IC-loop} und \textit{HS-gamma} gezeigt, dass die realen Evaluationsdaten auf eine ca. 5$m$ breiten und ca. 20$m$ bis 30$m$ langen Teilzone verteilt werden. Zusätzlich wurden die Orientierung der realen Evaluationsdaten als die Orientierung der dominierenden Aufnahmerichtung bestimmt. Die genannten Ergebnisse zeigen Parallele zu den Ergebnissen von \citet{acharyaBIMPoseNetIndoorCamera2019}, indem die von der Lokalisierung betroffenen Gebäudearealen eine vergleichbare Größe haben und die Orientierung überwiegend nur in einer Richtung bestimmt wird.

% PoseNet kann das
Grundsätzlich ist PoseNet weder auf einen ca. 5$m$ breiten und ca. 20$m$ bis 30$m$ langen Gebäudeareal noch für eine Orientierung begrenzt. \citet{walchImageBasedLocalizationUsing2017} konnten mit PoseNet durch das Trainieren und Evaluieren mit realen Daten in einem größeren Gebäudeareal als von den \textit{IC-loop} oder \textit{HS-gamma} Datensätzen mit einer Akkuratesse von 1.87$m$, 6.14° lokalisieren. Ebenso konnte in dieser Bachelorarbeit gezeigt werden, dass durch das Trainieren und Evaluieren mit synthetischen Daten eine durchschnittliche Akkuratesse 1.59$m$, 11.14° erzielt werden kann.


% Eventualitäten
Das Lokalisieren auf einem begrenzten Gebäudeareal in \textit{IC-loop} kann auf \textit{perceptual-aliasing} zurückgeführt werden, da sich die Innenräume der vertikal sowie horizontal verlaufende Strecken stark ähnelten. Im Gegensatz dazu waren in \textit{HS-gamma} die Innenräume der vertikal und entlang der Schlaufe verlaufenden Strecken von den horizontalen Strecken optisch differenzierbar. Dennoch wurden in \textit{HS-gamma} die realen Evaluationsdaten \textit{überwiegend} auf einen Bereich der linken horizontalen Strecke lokalisiert. Ebenso wurde die Orientierung als die dominierende Aufnahmerichtung bestimmt.


\citet{acharyaBIMPoseNetIndoorCamera2019} stellten überraschend fest, dass das zunehmende \textit{Level of Detail} (LOD) der Gebäudesimulation (\textit{Trainingsdaten}) bei der Evaluation mit den Gradientenbilder der realen Daten zur Abnahme der Akkuratesse führte. Angesichts des höheren LODs des HS-gamma Datensatzes liegt die Schlussfolgerung nahe, dass das die schlechte Akkuratesse im 

% datentypen
Außerdem erzielten \citet{acharyaBIMPoseNetIndoorCamera2019} ihre besten Ergebnisse durch das Trainieren mit den Gradientenbilder der synthetischen Kantenbilder (\textit{grad-edge}). In dieser Arbeit konnte häufiger mit den (\textit{grad-cartoon}) Datensätzen bessere Akkuratesse erzielt werden. Allerdings kann auf dieser Tatsache kein Typ der synthetischen Daten als Bestes festgelegt werden, da zum nötigen Vergleich aufgrund der begrenzten Anzahl des Trainingsprozesses das Vorkommen der bestmöglichen Akkuratesse von jedem Datentyp ausgeschlossen werden kann. 
 

% Beschränken
Die Akkuratesse eines KNNs ist in dieser Arbeit durch den stochastischen Gradietenabstiegsverfahren \textit{AdaGrad} bei der Optimierung des \textit{Losses} im Trainingsprozess vom Zufall abhängig. Die durch den Zufall bedingten bestmöglichen Akkuratesse zu finden würde den Rahmen dieser Bachelorarbeit aus Zeitgründen sprengen. Weiterführende Forschungen können bei gleichen Hyperparametern die Anzahl der Trainingsprozesse erhöhen, um bessere Ergebnisse zu erzielen oder auszuschließen sowie den Zusammenhang der Datentypen und den Ergebnissen untersuchen. 


Darüber hinaus behandelt diese Arbeit nicht die Optimierung der Hyperparameter. Die Hyperparameter wurden aus \citet{acharyaBIMPoseNetIndoorCamera2019} übernommen bzw. gleichermaßen bestimmt oder im selben Verhältnis zum Datensatz gewählt. Die Optimierung der Hyperparameter kann auf den jeweiligen Datensätzen zu besseren Ergebnissen führen und ist daher eine weitere Empfehlung für weiterführende Forschungen. 

In der Literatur wurden raumzeitliche Informationen aus Bildsequenzen von aufeinanderfolgenden Frames berücksichtigt, um das perceptual-aliasing Problem zu behandeln \cite{walchImageBasedLocalizationUsing2017, clarkVidLocDeepSpatioTemporal2017}. Weiterhin wurde PoseNet an einem Bayessian Neural Network angepasst, um die Unsicherheit der Ergebnisse zu modellieren. Dadurch ist es in der Evaluationsphase möglich Ergebnisse zu vertrauen oder zu verwerfen \cite{kendallModellingUncertaintyDeep2016}. 

% nochmal aufgreifen

Daher können weitere Forschungsprojekte die im Rahmen dieser Bachelorarbeit erhobenen Datensätze auf die PoseNet Nachfolger \cite{kendallModellingUncertaintyDeep2016, walchImageBasedLocalizationUsing2017, clarkVidLocDeepSpatioTemporal2017} anwenden und die Fähigkeit zur Lokalisierung untersuchen.