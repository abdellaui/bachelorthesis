% !TEX root = ../my_thesis.tex

\section{Diskussion}
\label{sec:kapitel_5}

% Zusammenfassung Ergebnisse
Die Ergebnisse der Untersuchungen haben gezeigt, dass der Ansatz von \citet{acharyaBIMPoseNetIndoorCamera2019}, worin das PoseNet mit Gradietenbilder der synthetischen Daten trainiert und mit Gradientenbildern der realen Daten evaluiert wird, ohne weiteren Regelungen auf die im Rahmen dieser Bachelorarbeit erhobenen Datensätze eine durchschnittliche Akkuratesse von ca. 8.45$m$ in der Position und 47.34° in der Orientierung erreichte. Die Evaluationen mit den synthetischen Gradientenbildern haben gezeigt, dass mit Daten aus der gleichen Domäne eine durchschnittliche Akkuratesse von 1.07$m$ in der Position und 8.34° in der Orientierung erzielt werden kann. 


In Gebäuden kommt es häufiger vor, dass unterschiedliche Stellen eines Innenraumes ähnliche Merkmale besitzen. Dieses Problem ist in der Literatur auch als \textit{perceptual-aliasing} bekannt und stellt einen der größten Herausforderungen von Lokalisierungsverfahren dar \cite{lowryVisualPlaceRecognition2016}.

Die Ergebnisse der Strecken \textit{HS-stairs-down} und \textit{HS-stairs-up} haben gezeigt, dass die Evaluationsdaten zwischen der oberen und unteren Treppenlauf lokalisiert wurden. Eine konkrete Unterscheidung zwischen den unteren und oberen Treppenlauf ist nicht zuerkennen. Dies kann auf perceptual-aliasing zurückgeführt werden, da die Treppenläufe sich stark ähneln. Weiterhin haben die Ergebnisse der Strecken \textit{IC-loop} und \textit{HS-gamma} gezeigt, dass die Evaluationsdaten auf eine ca. 20$m$ bis 30$m$ langen horizontalen Teilzone verteilt werden. In \textit{IC-loop} kann das Lokalisieren auf einem ca. 30$m$ langen Teilzone auf perceptual-aliasing zurückgeführt werden, da sich die Innenräume der vertikal sowie horizontal verlaufende Strecken stark ähnelten (s. Abb. \ref{fig:trajectories}). Hingegen waren in \textit{HS-gamma} die Innenräume der vertikal verlaufenden Strecke von den horizontal und entlang der Schleife verlaufenden Strecken differenzierbar. Dennoch werden in \textit{HS-gamma} die Evaluationsdaten nur auf ein ca. 20$m$ langen Teilzone der horizontalen Strecke lokalisiert.
% Acharyas länge war 20m, posenet bzw. poselstm konnten schon auf krasseres


Die Evaluationsdaten wiesen eine Abweichung (\textit{Drift}) bis zur 5\% (vgl. Abb. \ref{fig:trajectories}) auf.
%


% Beschränkung
% -Hyperparameter
% -Drift

% Empfehlung



Die Hyperparameter wurden überwiegen aus \citet{acharyaBIMPoseNetIndoorCamera2019} übernommen. Die Optimierung der Hyperparameter kann zu besseren Ergebnissen auf den Datensätzen führen und ist daher eine Empfehlung für weitere Forschung. Insbesondere können die im Rahmen dieser Bachelorarbeit erhobenen Datensätze auf die Nachfolger der PoseNet Architektur wie z.B. \cite{kendallModellingUncertaintyDeep2016, walchImageBasedLocalizationUsing2017, kendallGeometricLossFunctions2017, clarkVidLocDeepSpatioTemporal2017} angewandt werden.

 %Dieser Ansatz Potential eines des Experimentes zur Pose Bestimmung mit Daten derselben Domäne. 
% Interpretation

% 

