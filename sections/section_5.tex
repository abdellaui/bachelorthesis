% !TEX root = ../my_thesis.tex

\section{Diskussion}
\label{sec:kapitel_5}

% Zusammenfassung Ergebnisse
Die Ergebnisse der Untersuchungen haben gezeigt, dass der Ansatz von \citet{acharyaBIMPoseNetIndoorCamera2019}, worin das PoseNet mit Gradietenbilder der synthetischen Daten trainiert und mit Gradientenbildern der realen Daten evaluiert wird, ohne weiteren Regelungen auf die im Rahmen dieser Bachelorarbeit erhobenen Datensätze eine durchschnittliche Akkuratesse von ca. 8.45$m$ in der Position und 47.34° in der Orientierung erreichte. Die Evaluationen mit den synthetischen Gradientenbildern haben gezeigt, dass mit Daten aus der gleichen Domäne eine durchschnittliche Akkuratesse von 1.07$m$ in der Position und 8.34° in der Orientierung erzielt werden kann. 


In Gebäuden kommt es häufiger vor, dass unterschiedliche Stellen eines Innenraumes ähnliche Eigenschaften besitzen. Dies ist in der Literatur auch als perceptual- aliasing-Problem bekannt und stellt einen der größten Herausforderungen von Lokalisierungsansätzen dar \cite{lowryVisualPlaceRecognition2016}.
Lokalisierungsansätze kommen an ihre Grenze 
perceptual-aliasing Problem

Auf den Strecken \textit{IC-loop} und \textit{HS-gamma} wurden die Evaluationsdaten auf eine ca. 20$m$ bis 30$m$ Teilzone verteilt. Während in \textit{IC-loop} 


%


 %Dieser Ansatz Potential eines des Experimentes zur Pose Bestimmung mit Daten derselben Domäne. 
% Interpretation

% 

% -Hyperparameter
% -Drift

% Beschränkung

% Empfehlung

% Hyperparameter insbesondere Architektur etc