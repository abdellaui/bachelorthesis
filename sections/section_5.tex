% !TEX root = ../my_thesis.tex

\section{Diskussion}
\label{sec:kapitel_5}

% Zusammenfassung Ergebnisse
Die Ergebnisse der Untersuchungen haben gezeigt, dass der Ansatz von \citet{acharyaBIMPoseNetIndoorCamera2019}, worin das PoseNet mit Gradietenbilder der synthetischen Daten trainiert und mit Gradientenbildern der realen Daten evaluiert wird,  ohne weiteren Regelungen auf die im Rahmen dieser Bachelorarbeit erhobenen Datensätze eine durchschnittliche Akkuratesse von ca. 8.45$m$ in der Position und 47.34° in der Orientierung erzielte. Mit diesen Ergebnissen ist ein Lokalisierungsverfahren in Gebäuden undenkbar.

Die Evaluation mit den synthetischen Gradientenbildern ergab eine durchschnittliche Akkuratesse von 1.07$m$ in der Position und 8.34° in der Orientierung.

% Interpretation

% 

% -Hyperparameter
% -Drift

% Beschränkung

% Empfehlung

% Hyperparameter insbesondere Architektur etc