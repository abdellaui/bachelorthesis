% !TEX root = ../my_thesis.tex

\section{Diskussion}
\label{sec:kapitel_5}


% Zusammenfassung Ergebnisse
Die Ergebnisse der Untersuchungen haben gezeigt, dass der Ansatz von \citet{acharyaBIMPoseNetIndoorCamera2019}, worin das PoseNet mit Gradietenbilder der synthetischen Daten trainiert und mit Gradientenbildern der realen Daten evaluiert wird, ohne Optimierung der Hyperparameter auf die im Rahmen dieser Bachelorarbeit erhobenen Datensätze eine durchschnittliche Akkuratesse von 8.45$m$ in der Position und 47.34° in der Orientierung erreichte. Die Evaluationen mit den synthetischen Gradientenbildern haben gezeigt, dass mit Daten aus der gleichen Domäne eine durchschnittliche Akkuratesse von 1.07$m$ in der Position und 8.34° in der Orientierung erzielt werden kann. 


% Perceptual aliasing & treppen
In Gebäuden kommt es häufiger vor, dass unterschiedliche Stellen eines Innenraumes ähnliche Merkmale besitzen und somit schwer voneinander zu unterscheiden sind. Dieses Problem ist in der Literatur auch als \textit{perceptual-aliasing} bekannt und stellt einen der größten Herausforderungen von Lokalisierungsverfahren dar \cite{lowryVisualPlaceRecognition2016}. Auf den \textit{HS-stairs-down} und \textit{HS-stairs-up} Strecken wurden die realen Evaluationsdaten grundsätzlich zwischen der oberen und unteren Treppenlauf lokalisiert. Eine Generalisierungsfähigkeit zwischen Oben und Unten der KNNs ist nicht zuerkennen. Vielmehr ist eine \textit{unsichere} (zufällige) Zuordnung der KNNs festzuhalten. Dies kann auf perceptual-aliasing zurückgeführt werden und erklärt die abwechselnde Fehlerrate auf den Evaluationsdaten der Treppenläufe.

% loop und gamma auf fenster aufmerksam machen
Weiterhin haben die Ergebnisse der Strecken \textit{IC-loop} und \textit{HS-gamma} gezeigt, dass die Evaluationsdaten auf eine ca. 5$m$ hohen und ca. 20$m$ bis 30$m$ langen Teilzone verteilt werden. Außerdem wurden die Orientierung der Evaluationsdaten als die Orientierung der dominierenden Aufnahmerichtung bestimmt. Diese Ergebnisse passen soweit überein mit den Ergebnissen von \citet{acharyaBIMPoseNetIndoorCamera2019}, dass die von der Lokalisierung betroffenen Gebäudearealen eine vergleichbare Größenordnungen haben und die Orientierung überwiegend in einer Richtung bestimmt wird.

In \textit{IC-loop} kann das Lokalisieren auf einem begrenzten Gebäudeareal auf perceptual-aliasing zurückgeführt werden, da sich die Innenräume der vertikal sowie horizontal verlaufende Strecken stark ähnelten. Im Gegensatz dazu waren in \textit{HS-gamma} die Innenräume der vertikal und entlang der Schleife verlaufenden Strecken von den horizontalen Strecken optisch differenzierbar. Dennoch wurden in \textit{HS-gamma} die Evaluationsdaten \textit{überwiegend} auf einen Bereich der linken horizontalen Strecke lokalisiert. 

%Aufgrund der optischen Ähnlichkeit (perceptual-aliasing) der horizontalen Strecken konnte das KNN zwischen der linken und rechten horizontalen Strecken nicht generalisieren. Daher wiesen die Evaluationsdaten der rechten horizontalen Strecke die größten Positionsfehler auf. Ebenso konnte das KNN zwischen der vertikalen und horizontalen Strecken nicht unterscheiden.

% STRECKE VON ACHARYA

PoseNet ist nicht begrenzt für die Bestimmung der dominierenden Aufnahmerichtung der Trainingsdaten und auf einen ca. 5$m$ hohen und ca. 20$m$ bis 30$m$ langen Gebäudeareal. \citet{walchImageBasedLocalizationUsing2017} konnten mit PoseNet durch das Trainieren und Evaluieren mit realen Daten in einem größeren Gebäudeareal als von den \textit{IC-loop} oder \textit{HS-gamma} Datensätzen mit einer Akkuratesse von 1.87$m$ in der Position und 6.14° in der Orientierung lokalisieren. Ebenso konnte in dieser Bachelorarbeit gezeigt werden, dass durch das Trainieren und Evaluieren mit synthetischen Daten eine durchschnittliche Akkuratesse von 1.07$m$ in der Position und 8.34° in der Orientierung erzielt werden kann.




%Allerdings ist die betroffene Korridorfläche von \citet{acharyaBIMPoseNetIndoorCamera2019} in einer vergleichbaren Größenordnung der Teilbereiche von den Evaluationen der \textit{IC-loop} oder \textit{HS-gamma} Strecken.






% Beschränken

Ziel dieser Arbeit ist den Ansatz von \citet{acharyaBIMPoseNetIndoorCamera2019} auf den erhobenen Datensätzen zu untersuchen. Dabei wird eine Lokalisierungsgenauigkeit im Meterbereich angestrebt, da PLACEHOLDER . Daher ist die Abweichung (\textit{Drift}) bis zu 5\% der realen Evaluationsdaten von den synthetischen Trainingsdaten unproblematisch. In Anbetracht der Ergebnisse ist eine Lokalisierungsverfahren ohne weitere Maßnahmen des Ansatzes von \citet{acharyaBIMPoseNetIndoorCamera2019} auf den in dieser Arbeit erhobenen Datensätzen auszuschließen.


Die Akkuratesse eines KNNs ist in dieser Arbeit durch den stochastischen Gradietenabstiegsverfahren \textit{AdaGrad} bei der Optimierung des \textit{Losses} im Trainingsprozess vom Zufall abhängig. Die durch den Zufall bedingten bestmöglichen Akkuratesse zu finden würde den Rahmen dieser Bachelorarbeit aus Zeitgründen sprengen. Deshalb wurden die Trainingsprozesse jeweils für die Strecken und Datentypen 5-mal wiederholt und die Netzwerke mit den besten Akkuratesse behalten. Weiterführende Forschungen können bei gleichen Hyperparametern die Anzahl der Trainingsprozesse erhöhen, um bessere Ergebnisse zu erzielen oder auszuschließen. 


Diese Arbeit behandelt nicht die Optimierung der Hyperparameter. Die Hyperparameter wurden aus \citet{acharyaBIMPoseNetIndoorCamera2019} übernommen bzw. gleichermaßen bestimmt oder im selben Verhältnis zum Datensatz gewählt. Die Optimierung der Hyperparameter kann auf den jeweiligen Datensätzen zu besseren Ergebnissen führen und ist daher eine weitere Empfehlung für weiterführende Forschungen. 

In der Literatur wurden raumzeitliche Informationen aus Bildsequenzen von aufeinanderfolgenden Frames berücksichtigt, um das perceptual-aliasing Problem zu behandeln \cite{walchImageBasedLocalizationUsing2017, clarkVidLocDeepSpatioTemporal2017}. Außerdem wurde PoseNet an einem Bayessian Neural Network angepasst, um die Unsicherheit der Ergebnisse zu modellieren. Dadurch ist es möglich in der Evaluationsphase möglich Ergebnisse zu vertrauen oder zu verwerfen \cite{kendallModellingUncertaintyDeep2016}. 

% nochmal aufgreifen

Daher können weitere Forschungsprojekte die im Rahmen dieser Bachelorarbeit erhobenen Datensätze auf die PoseNet Nachfolger \cite{kendallModellingUncertaintyDeep2016, walchImageBasedLocalizationUsing2017, clarkVidLocDeepSpatioTemporal2017} anwenden und die Fähigkeit zur Lokalisierung untersuchen.