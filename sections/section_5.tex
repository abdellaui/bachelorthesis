% !TEX root = ../my_thesis.tex

\section{Diskussion}
\label{sec:kapitel_5}


% Zusammenfassung Ergebnisse
Die Ergebnisse der Untersuchungen haben gezeigt, dass der Ansatz von \citet{acharyaBIMPoseNetIndoorCamera2019}, worin das PoseNet mit Gradietenbilder der synthetischen Daten trainiert und mit Gradientenbildern der realen Daten evaluiert wird, ohne Optimierung der Hyperparameter auf die im Rahmen dieser Bachelorarbeit erhobenen Datensätze eine durchschnittliche Akkuratesse von 8.45$m$ in der Position und 47.34° in der Orientierung erreichte. Die Evaluationen mit den synthetischen Gradientenbildern haben gezeigt, dass mit Daten aus der gleichen Domäne eine durchschnittliche Akkuratesse von 1.07$m$ in der Position und 8.34° in der Orientierung erzielt werden kann. 

In Gebäuden kommt es häufiger vor, dass unterschiedliche Stellen eines Innenraumes ähnliche Merkmale besitzen und somit schwer voneinander zu unterscheiden sind. Dieses Problem ist in der Literatur auch als \textit{perceptual-aliasing} bekannt und stellt einen der größten Herausforderungen von Lokalisierungsverfahren dar \cite{lowryVisualPlaceRecognition2016}. Auf den \textit{HS-stairs-down} und \textit{HS-stairs-up} Strecken wurden die realen Evaluationsdaten grundsätzlich zwischen der oberen und unteren Treppenlauf lokalisiert. Eine Generalisierungsfähigkeit zwischen Oben und Unten der KNNs ist nicht zuerkennen. Vielmehr ist eine \textit{unsichere} (zufällige) Zuordnung der KNNs festzuhalten. Dies kann auf perceptual-aliasing zurückgeführt werden und erklärt die abwechselnde Fehlerrate auf den Evaluationsdaten der Treppenläufe.


Weiterhin haben die Ergebnisse der Strecken \textit{IC-loop} und \textit{HS-gamma} gezeigt, dass die Evaluationsdaten auf eine ca. 5$m$ breiten und ca. 20$m$ bis 30$m$ langen Teilzone verteilt werden. In \textit{IC-loop} kann das Lokalisieren auf einem begrenzten Gebäudeareal auf perceptual-aliasing zurückgeführt werden, da sich die Innenräume der vertikal sowie horizontal verlaufende Strecken stark ähnelten. Im Gegensatz dazu waren in \textit{HS-gamma} die Innenräume der vertikal verlaufenden Strecke von den horizontal und entlang der Schleife verlaufenden Strecken differenzierbar. Dennoch werden in \textit{HS-gamma} die Evaluationsdaten \textit{nur} auf einen Bereich der horizontalen Strecke lokalisiert.

% eine Akkuratesse von 1.87$m$ in der Position und 47.34° in der Orientierung

\citet{walchImageBasedLocalizationUsing2017} konnten mit PoseNet durch das Trainieren und Evaluieren mit realen Daten in einem größeren Gebäudeareal als von den \textit{IC-loop} oder \textit{HS-gamma} Datensätzen mit einer Akkuratesse von 1.87$m$ in der Position und 6.14° in der Orientierung lokalisieren. Ebenso konnte in dieser Bachelorarbeit gezeigt werden, dass durch das Trainieren und Evaluieren mit synthetischen Daten eine durchschnittliche Akkuratesse von 1.07$m$ in der Position und 8.34° in der Orientierung erzielt werden kann. Daher ist ein Lokalisierungsverfahren über PoseNet nicht begrenzt auf einen ca. 5$m$ breiten und ca. 20$m$ bis 30$m$ langen Gebäudeareal. 




Allerdings ist die betroffene Korridorfläche von \citet{acharyaBIMPoseNetIndoorCamera2019} in einer vergleichbaren Größenordnung der Teilbereiche von den Evaluationen der \textit{IC-loop} oder \textit{HS-gamma} Strecken.







%Die Evaluationsdaten wiesen eine Abweichung (\textit{Drift}) bis zur 5\% auf (s. Abb. \ref{subsec:record_real_data}). Allerdings sind die Ergebnisse weitmerht
%



% Beschränkung
% -Hyperparameter
% -Drift

% Empfehlung


Die zeitintensiven Trainingsprozesse wurden aus Zeitgründen 5-mal wiederholt. Daher ist eine bessere Akkuratesse der in dieser Arbeit evaluierten Netzwerken kann nicht ausgeschlossen werden, da der Trainingsprozess durch den Loss-Optimierer AdaGrad einem Zufall unterworfen ist.

Die Hyperparameter wurden überwiegen aus \citet{acharyaBIMPoseNetIndoorCamera2019} übernommen. Die Optimierung der Hyperparameter kann zu besseren Ergebnissen auf den Datensätzen führen und ist daher eine Empfehlung für weitere Forschung. 
Um das perceptual-aliasing Problem zu behandeln, wurden in der Literatur raumzeitliche Information aus Bildsequenzen von aufeinanderfolgenden Frames berücksichtigt \cite{walchImageBasedLocalizationUsing2017, clarkVidLocDeepSpatioTemporal2017}. Außerdem wurde PoseNet an einem Bayessian Neural Network angepasst, um die Unsicherheit der Ergebnisse zu modellieren. Dadurch ist es möglich in der Evaluationsphase möglich Ergebnisse zu vertrauen oder zu verwerfen \cite{kendallModellingUncertaintyDeep2016}. Daher können die im Rahmen dieser Bachelorarbeit erhobenen Datensätze auf die Nachfolger der PoseNet Architektur wie z.B. \cite{kendallModellingUncertaintyDeep2016, walchImageBasedLocalizationUsing2017, clarkVidLocDeepSpatioTemporal2017} angewendet werden.

 %Dieser Ansatz Potential eines des Experimentes zur Pose Bestimmung mit Daten derselben Domäne. 
% Interpretation

% 

