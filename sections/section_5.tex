% !TEX root = ../my_thesis.tex

\section{Diskussion}
\label{sec:kapitel_5}

\subsection{Diskussion der Methode}
% drift
% hyperparameter
% zufall
\subsection{Diskussion der Ergebnisse}

%Angesichts dieser Ergebnisse liegt die Schlussfolgerung nahe, dass das Evaluieren mit den Gradientenbilder der realen Daten auf das mit synthetischen Daten trainierten PoseNet.
%Es ist nicht das Ziel dieser Arbeit den visuellen Lokalisierungsansatz von \citet{acharyaBIMPoseNetIndoorCamera2019} auf den erhobenen Datensätzen zu untersuchen. Dabei wird eine Lokalisierungsgenauigkeit im Meterbereich angestrebt, da PLACEHOLDER . Daher ist die Abweichung (\textit{Drift}) bis zu 5\% der realen Evaluationsdaten von den synthetischen Trainingsdaten unproblematisch. In Anbetracht der Ergebnisse ist eine Lokalisierungsverfahren ohne weitere Maßnahmen des Ansatzes von \citet{acharyaBIMPoseNetIndoorCamera2019} auf den in dieser Arbeit erhobenen Datensätzen auszuschließen.


In der vorliegenden Bachelorarbeit konnte gezeigt werden, dass der Ansatz von \citet{acharyaBIMPoseNetIndoorCamera2019} ohne eine Optimierung der Hyperparameter eine durchschnittliche Akkuratesse von 10.95$m$ in der Position und 49.69° in der Orientierung erreichte. Die Evaluationen mit den synthetischen Gradientenbildern konnten zeigen, dass eine durchschnittliche Akkuratesse von 1.19$m$, 8.35° mit synthetischen Daten erzielt werden konnte.


%übereinstimmend mit den Referenzwerten von 1.17$m$, 7.34° erziel
% TODO: REFERENZWERT%
% würde reichen für

% Perceptual aliasing & treppen
In Gebäuden kommt es häufiger vor, dass unterschiedliche Stellen eines Innenraumes ähnliche Merkmale besitzen und somit schwer voneinander zu unterscheiden sind. Dieses Problem ist in der Literatur auch als \textit{perceptual-aliasing} bekannt und stellt eine der größten Herausforderungen von Lokalisierungsverfahren dar \cite{lowryVisualPlaceRecognition2016}. In dieser Arbeit wurde diese Herausforderung durch das Anstreben einer domänenübergreifenden Abstraktion bzw. das Erlenen von den Merkmalen der realen Daten aus den simulierten Daten zusätzlich verstärkt.


Die realen Evaluationsdaten auf den \textit{HS-stairs-down} und \textit{HS-stairs-up} Strecken wurden grundsätzlich zwischen der oberen und unteren Treppenlauf lokalisiert. Eine Generalisierungsfähigkeit zwischen Oben und Unten der KNNs war nicht erkennbar. Vielmehr konnte eine \textit{zufällige} Zuordnung der KNNs interpretiert werden. Dies könnte aufgrund von wiederholenden Strukturen einer Treppe auf \textit{perceptual-aliasing} zurückgeführt werden und würde die abwechselnde Fehlerrate auf den Evaluationsdaten der Treppenläufe erklären.
% loop und gamma auf muster aufmerksam machen  & STRECKE VON ACHARYA
Außerdem zeigten die Ergebnisse der Strecken \textit{IC-loop} und \textit{HS-gamma} eine Verteilung der realen Evaluationsdaten auf eine ca. 5m breiten und ca. 20m bis 30m langen Teilzone (vgl. Abb. \ref{subfig:ic_fig2, subfig:hs_gamma_fig2}). Zusätzlich wurde die Orientierung der realen Evaluationsdaten als die Richtung der von der Anzahl her überwiegenden Trainingsdaten bestimmt.


Die Ergebnisse von \textit{IC-loop} und \textit{HS-gamma} zeigten Parallelen zu den Ergebnissen von \citet{acharyaBIMPoseNetIndoorCamera2019} bzgl. der mit den Lokalisierungsergebnissen enthaltenden Gebäudearealen und 
der überwiegend in eine Richtung bestimmte Orientierung.
% Eventualitäten
Das Lokalisieren aller realen \textit{IC-loop} Evaluationsdaten auf einem begrenzten Gebäudeareal könnte auf \textit{perceptual-aliasing} zurückgeführt werden, da sich die Innenräume der vertikal sowie horizontal verlaufenden Strecken optisch stark ähnelten. Im Gegensatz dazu waren in \textit{HS-gamma} die Innenräume der vertikal und entlang der Schlaufe verlaufenden Strecken von den horizontalen Strecken optisch differenzierbar. Dennoch wurden in \textit{HS-gamma} ebenfalls die realen Evaluationsdaten überwiegend auf einem Bereich der linken horizontalen Strecke lokalisiert. Obwohl einige Evaluationsdaten nahe der Schlaufe lokalisiert werden konnten, wurde dennoch die Orientierung als die Richtung der horizontalen Strecken bestimmt (s. Abb. \ref{subfig:hs_gamma_fig2}, \ref{subfig:hs_gamma_fig4}). Hierbei ist über \textit{perceptual-aliasing} eine Erklärung der Ergebnisse nicht möglich.

\citet{acharyaBIMPoseNetIndoorCamera2019} stellten überraschend fest, dass die Akkuratesse mit einem zunehmendem \textit{Level of Detail} (LOD) der Gebäudesimulation (\textit{Trainingsdaten}) bei der Evaluation mit den Gradientenbildern der realen Daten abnahm. Zeitgleich hatten dieselben Autoren ein hohes LOD für eine bessere Akkuratesse empfohlen \cite{acharyaBIMPoseNetIndoorCamera2019}. Angesichts dieser Feststellung könnte das hohe LOD der HS-Gebäudesimulation für die unzureichende Generalisierungsfähigkeit des KNNs auf den HS-Datensätzen möglicherweise erklären. Allerdings zeigte die Evaluation der \textit{HS-gamma} Strecke falsche Ergebnisse weit über eine vorstellbare Wirkung des LODs. Ebenso sollte nicht außer Acht gelassen werden, dass die Ergebnisse von \textit{HS-gamma} Gemeinsamkeiten zu den Ergebnissen von \textit{IC-loop} und der von \citet{acharyaBIMPoseNetIndoorCamera2019} nachweisen.
%TODO: frage, why got parrarels

%Angesichts dieser Parallelen und das Fehlen einer LOD Grenze
%Eine Grenze des LODs geht aus nicht hervor, 
% Dies erklärt jedoch nicht die Gemeinsamkeiten der ergebniise.

%Allerdings ist die schlechte Generalisierungsfähigkeit 


%Allerdings liegt angesichts der oben genannten Parallelen der Ergebnisse die Schlussfolgerung näher, dass über PoseNet bei gleichen Hyperparametern wie in \citet{acharyaBIMPoseNetIndoorCamera2019} begrenzt auf das Erlenen einer Richtung in einem kleinen Gebäudeareal ist.





% PoseNet kann das
Die Gemeinsamkeiten der Ergebnisse wirkten jedoch fraglich, da PoseNet grundsätzlich weder auf einen ca. 5$m$ breiten und ca. 20$m$ bis 30$m$ langen Gebäudeareal noch für eine einzige Orientierung begrenzt war. Zudem konnte in dieser Bachelorarbeit gezeigt werden, dass durch das Trainieren und Evaluieren mit den hier erhobenen realen Datensätzen eine durchschnittliche Akkuratesse von 1.17$m$, 7.34° und mit den synthetischen Datensätzen eine durchschnittliche Akkuratesse 1.19$m$, 8.35° erzielt werden konnte. Ferner konnten \citet{walchImageBasedLocalizationUsing2017} mit PoseNet durch das Trainieren und Evaluieren mit realen Daten in einem größeren Gebäudeareal als von den \textit{IC-loop} oder \textit{HS-gamma} Datensätzen mit einer Akkuratesse von 1.87$m$, 6.14° die Pose bestimmen. Daher liegt angesichts der oben genannten Parallelen der Ergebnisse die Schlussfolgerung nahe, dass PoseNet bei gleichen Hyperparametern wie in \cite{acharyaBIMPoseNetIndoorCamera2019} mit den Gradientenbildern der simulierten Daten für die Evaluation mit den Gradientenbildern der realen Daten auf einem begrenzten Gebäudeareal nur in eine Richtung trainiert werden kann.


% TODO: annahme machen




 






\subsection{Ausblick}

% Beschränken
Die Akkuratesse eines KNNs ist in dieser Arbeit durch den stochastischen Gradietenabstiegsverfahren \textit{AdaGrad} bei der Optimierung des \textit{Losses} im Trainingsprozess vom Zufall abhängig. Die durch den Zufall bedingte bestmögliche Akkuratesse zu finden würde aus Zeitgründen den Rahmen dieser Bachelorarbeit sprengen. Weiterführende Forschungen können bei gleichen Hyperparametern die Anzahl der Trainingsprozesse erhöhen, um bessere Ergebnisse zu erzielen oder auszuschließen.

%datentypen
Weiterhin erzielten \citet{acharyaBIMPoseNetIndoorCamera2019} ihre besten Ergebnisse durch das Trainieren mit den Gradientenbilder der synthetischen Kantenbilder (\textit{grad-edge}). In dieser Arbeit konnte eine bessere Akkuratesse mit den \textit{grad-cartoon} Datensätzen häufiger erzielt werden (s. Tab. \ref{tab:results_ic,tab:results_hs_gamma, tab:results_hs_stairs_up, tab:results_hs_stairs_down}). Dennoch kann auf dieser Tatsache kein synthetischer Datentyp als der Beste festgelegt werden, da hierzu die bestmögliche Akkuratesse von jedem Datentyp weder sichergestellt noch ausgeschlossen werden konnte.


Darüber hinaus wurde die Optimierung der Hyperparameter in dieser vorliegenden Arbeit nicht behandelt. Die Hyperparameter wurden aus \citet{acharyaBIMPoseNetIndoorCamera2019} übernommen bzw. gleichermaßen bestimmt oder im selben Verhältnis zum Datensatz gewählt. Die Optimierung der Hyperparameter kann auf den jeweiligen Datensätzen zu besseren Ergebnissen führen und ist daher eine weitere Empfehlung zu weiterführenden Forschungen. 

Um das \textit{perceptual-aliasing} Problem zu behandeln, wurde in der Literatur raumzeitliche Informationen aus Bildsequenzen von aufeinanderfolgenden Frames berücksichtigt \cite{walchImageBasedLocalizationUsing2017, clarkVidLocDeepSpatioTemporal2017}. Des Weiteren wurde PoseNet an einem Bayessian Neural Network angepasst, um die Unsicherheit der Ergebnisse zu modellieren. Dadurch ist es in der Evaluationsphase möglich, den Ergebnisse zu vertrauen oder zu verwerfen \cite{kendallModellingUncertaintyDeep2016}. Angesichts dessen können weitere Forschungsprojekte die hier erhobenen Datensätze auf die Nachfolger von PoseNet \cite{kendallModellingUncertaintyDeep2016, walchImageBasedLocalizationUsing2017, clarkVidLocDeepSpatioTemporal2017} anwenden und die Fähigkeit zur Lokalisierung untersuchen.