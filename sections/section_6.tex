% !TEX root = ../my_thesis.tex

\section{Fazit}
\label{sec:kapitel_6}
Das Ziel dieser Bachelorarbeit war es, den Ansatz zur Pose Estimation anhand von Convolutional Neuroal Networks und simulierten 3D Daten von \citet{acharyaBIMPoseNetIndoorCamera2019} in größeren Gebäudesimulationen und auf \textit{längeren} Strecken zu untersuchen.
Insgesamt wurde der Ansatz auf vier Strecken in zwei unterschiedlichen Gebäudesimulationen untersucht. 

Zusammenfassend konnte durch die Untersuchung festgestellt werden, dass ein Lokalisierungsverfahren über das Convolutional Neuroal Network PoseNet durch das Tranieren bei übereinstimmenden Hyperparametern wie in \cite{acharyaBIMPoseNetIndoorCamera2019} von den hier erhobenen synthetischen Daten nicht möglich ist. 

PoseNet konnte auf den Datensätzen mit einer durchschnittlichen Akkuratesse von 10.95$m$ in der Position und 49.69° in der Orientierung auf einem begrenzten Gebäudeareal in nur einer Richtung trainiert werden. Angesichts der Gemeinsamkeiten dieser Ergebnisse und den von \citet{acharyaBIMPoseNetIndoorCamera2019} liegt die Schlussfolgerung nahe, dass PostNet bei gleichen Hyperparametern wie in \cite{acharyaBIMPoseNetIndoorCamera2019} mit den Gradientenbildern der simulierten Daten für die Evaluation mit den Gradientenbildern der realen Daten auf einem begrenzten Gebäudeareal nur in eine Richtung trainiert werden kann.