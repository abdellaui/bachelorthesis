% !TEX root = ../my_thesis.tex

\section{Fazit}
\label{sec:kapitel_6}
Das Zeil dieser Bachelorarbeit war es, den Ansatz von \citet{acharyaBIMPoseNetIndoorCamera2019} in größeren Gebäudesimulationen und auf \textit{längeren} Strecken zu untersuchen.
Zusammenfassend konnte in der vorliegenden Arbeit festgestellt werden, dass ein Lokalisierungsverfahren über PoseNet durch das Tranieren bei übereinstimmenden Hyperparametern wie in \cite{acharyaBIMPoseNetIndoorCamera2019} von den hier erhobenen synthetischen Daten 


 dass das KNN PostNet keinen adäquaten Ersatz zur Erhebung der realen Daten durch synthetische Daten ermöglichen konnte. PostNet konnte die Akkurattessen mit ... Fehlerrate und in einem begrenzten Areal in einer Richtung erheben. Damit liegt die Schlussfolgerung nahe, dass PostNet die sehr zeit- und kostenaufwändige reale Datenerhebung nicht mit einer synthetischen Datenerhebung ersetzen kann.
