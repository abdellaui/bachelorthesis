% !TEX root = ../my_thesis.tex

\section{Fazit}
\label{sec:kapitel_6}
Das Ziel dieser Bachelorarbeit war, den Ansatz von \citet{acharyaBIMPoseNetIndoorCamera2019} zur Pose Estimation anhand von Convolutional Neural Networks und simulierten 3D-Daten in größeren Gebäudesimulationen und auf längeren Strecken zu untersuchen.
Insgesamt wurde der Ansatz in zwei unterschiedlichen Gebäuden auf vier Strecken untersucht. Zusammenfassend konnte durch die Untersuchungen festgestellt werden, dass ein Lokalisierungsverfahren über den Ansatz von \citet{acharyaBIMPoseNetIndoorCamera2019} mit den hier erhobenen Datensätzen nicht möglich ist.


Eine Akkuratesse von ca. 1$m$ in der Position und 8° in der Orientierung konnte durch die zugrundeliegende CNN PoseNet mit Daten der gleichen Domäne in der Trainings- und Evaluationsphase erzielt werden. Im Gegensatz dazu konnte bei domänenübergreifender Anwendung eine durchschnittlichen Akkuratesse von 10.95$m$, 49.69° erreicht werden. Hierbei konnte PoseNet nur auf einem begrenzten Gebäudeareal in einer Richtung trainiert werden. Dies zeigte Parallelen zu den Ergebnissen von \citet{acharyaBIMPoseNetIndoorCamera2019}, da diese PoseNet auf einer überwiegend in eine Richtung verlaufenden Strecke in einem kleinem Korridor trainierten. Angesichts dieser Parallelen lag die Schlussfolgerung nahe, dass durch das Trainieren mit den Gradientenbildern der synthetischen Daten bei gleichen Hyperparametern wie in \cite{acharyaBIMPoseNetIndoorCamera2019} von PoseNet die Gradientenbilder der realen Evaluationsdaten auf einem begrenzten Gebäudeareal nur in eine Richtung bestimmt werden können.

% Motivieren
Ein Lokalisierungsansatz innerhalb von Gebäuden mit einer Positionsakkuratesse von ca. 11$m$ ist undenkbar. Weiterhin ist eine potenzielle Positionsakkuratesse von 1$m$ im direkten Gebrauch für Anwendungen wie die automatische Bauforstschritterfassung sowie das Facility-Manage\-ment und die Navigation über Augmented Reality unzureichend. Hierbei könnte allerdings die Pose mit einem Kaskadeneffekt z.B. über weiterführenden Netzwerken oder modellbasierten Ansätzen lokal korrigiert werden.
% OUTLOOK

Die Abnahme der Akkuratesse bei domänenübergreifender Anwendung von PoseNet lassen sich auf die strukturellen Unterschiede der Realität zur Simulationsumgebung wie z.B. das Fehlen von diversen Objekten und die Existenz von domänenspezifischen Artefakten zurückführen (vgl. \cite{acharyaBIMPoseNetIndoorCamera2019}). Es wäre in diesem Zusammenhang lohnenswert zu untersuchen, ob die Diskrepanzminimierung der synthetischen Datensätze zu den realen Datensätzen wie z.B. durch \textit{Generative Adversarial Networks} zu einem Fortschritt führt. Ferner könnte ein Fortschritt durch Beschränkungen der möglichen Posen im Trainingsprozess erzielt werden, sodass die Positionsergebnisse immer in den betroffenen Innenräumen der Gebäuden liegen.
