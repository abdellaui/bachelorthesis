% !TEX root = ../my_thesis.tex

\section{Fazit}
\label{sec:kapitel_6}
Das Ziel dieser Bachelorarbeit war, den Ansatz zur Pose Estimation anhand von Convolutional Neural Networks und simulierten 3D Daten von \citet{acharyaBIMPoseNetIndoorCamera2019} in größeren Gebäudesimulationen und auf längeren Strecken zu untersuchen.
Insgesamt wurde der Ansatz auf vier Strecken in zwei unterschiedlichen Gebäudesimulationen untersucht. Zusammenfassend konnte durch die Untersuchung festgestellt werden, dass ein Lokalisierungsverfahren in den betroffenen Gebäuden über das Convolutional Neuroal Network PoseNet durch das Trainieren bei übereinstimmenden Hyperparametern wie in \cite{acharyaBIMPoseNetIndoorCamera2019} mit den hier erhobenen synthetischen Daten nicht möglich ist. 

PoseNet konnte auf den Datensätzen mit einer durchschnittlichen Akkuratesse von 10.95$m$ in der Position und 49.69° in der Orientierung auf einem begrenzten Gebäudeareal in nur einer Richtung trainiert werden. Angesichts der Parallelen dieser Ergebnisse und den von \citet{acharyaBIMPoseNetIndoorCamera2019} liegt die Schlussfolgerung nahe, dass durch das Trainieren mit den Gradientenbildern der simulierten Daten bei gleichen Hyperparametern wie in \cite{acharyaBIMPoseNetIndoorCamera2019} von PoseNet die Gradientenbilder der realen Evaluationsdaten auf einem begrenzten Gebäudeareal nur in eine Richtung bestimmt werden können.

//HIER FEHLT NOCH IWAS