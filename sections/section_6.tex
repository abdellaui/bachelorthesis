% !TEX root = ../my_thesis.tex

\section{Fazit}
\label{sec:kapitel_6}
Das Zeil dieser Bachelorarbeit war es, den Ansatz zur Pose Estimation anhand von Convolutional Neuroal Networks und simulierten 3D Daten von \citet{acharyaBIMPoseNetIndoorCamera2019} in größeren Gebäudesimulationen und auf \textit{längeren} Strecken zu untersuchen.
Insgesamt wurde der Ansatz mit vier Datensätze untersucht. Zwei der Datensätze verliefen auf einer ebenen Strecke und die zwei Weiteren verliefen entlang einer Treppe.

Zusammenfassend konnte in der vorliegenden Arbeit festgestellt werden, dass ein Lokalisierungsverfahren über das Convolutional Neuroal Network PoseNet durch das Tranieren bei übereinstimmenden Hyperparametern wie in \cite{acharyaBIMPoseNetIndoorCamera2019} von den hier erhobenen synthetischen Daten nicht möglich ist. 

PoseNet konnte mit einer durchschnittlichen Akkuratesse von 10.95$m$ in der Position und 49.69° in der Orientierung auf einem begrenzten Gebäudeareal in einer Richtung trainiert werden. Damit liegt die Schlussfolgerung nahe, dass PostNet bei der 

