% !TEX root = ../my_thesis.tex

\section{Fazit}
\label{sec:kapitel_6}
Das Ziel dieser Bachelorarbeit war, den Ansatz von \citet{acharyaBIMPoseNetIndoorCamera2019} zur Pose Estimation anhand von Convolutional Neural Networks und simulierten 3D-Daten in größeren Gebäudesimulationen und auf längeren Strecken zu untersuchen.
Insgesamt wurde der Ansatz in zwei unterschiedlichen Gebäuden auf vier Strecken untersucht. Zusammenfassend konnte durch die Untersuchungen festgestellt werden, dass ein Lokalisierungsverfahren über den Ansatz von \citet{acharyaBIMPoseNetIndoorCamera2019} mit den hier erhobenen Datensätzen nicht möglich ist.


Mit der zugrundeliegende CNN PoseNet konnte eine durchschnittliche Akkuratesse bei Daten der gleichen Domäne in der Trainings- sowie Evaluationsphase eine durchschnittlichen Akkuratesse von 1.18$m$ in der Position und 7.34° in der Orientierung erzielt werden. Im Vergleich dazu konnte PoseNet bei domänenübergreifenden Nutzung die Pose mit einer durchschnittlichen Akkuratesse von 10.95$m$ , 49.69° bestimmen. Außerdem konnte PoseNet nur auf einem begrenzten Gebäudeareal in einer Richtung trainiert werden. Dies zeigte Parallelen zu den Ergebnisse von \citet{acharyaBIMPoseNetIndoorCamera2019}, da diese PoseNet auf einer überwiegend in eine Richtung verlaufenden Strecke in einem kleinem Korridor trainierten. Angesichts dieser Parallelen lag die Schlussfolgerung nahe, dass durch das Trainieren mit den Gradientenbildern der simulierten Daten bei gleichen Hyperparametern wie in \cite{acharyaBIMPoseNetIndoorCamera2019} von PoseNet die Gradientenbilder der realen Evaluationsdaten auf einem begrenzten Gebäudeareal nur in eine Richtung bestimmt werden können.


%\subsection{Empfehlungen für weiterführende Forschungen}

%datentypen
\citet{acharyaBIMPoseNetIndoorCamera2019} erzielten ihre besten Ergebnisse durch das Trainieren mit den Gradientenbildern der synthetischen Kantenbilder (\textit{grad-edge}). In dieser Arbeit konnte häufiger eine bessere Akkuratesse mit den \textit{grad-cartoon} Datensätzen erzielt werden (s. Tab. \ref{tab:results_ic}- \ref{tab:results_hs_stairs_down}). Dennoch kann auf dieser Tatsache kein synthetischer Datentyp als der Beste festgelegt werden, da hierzu die bestmögliche Akkuratesse von jedem Datentyp weder sichergestellt noch ausgeschlossen werden konnte. Das wird wie in \ref{subsec:disc_methode} erwähnt dadurch begründet, dass die Akkuratesse vom Zufall abhängig ist. Daher können weiterführende Forschungen bei gleichen Hyperparametern die Anzahl der Trainingsprozesse erhöhen, um bessere Ergebnisse zu erzielen oder auszuschließen. Infolgedessen könnte der Zusammenhang zwischen den Ergebnissen und den Datentypen untersucht werden. 

Außerdem wurde die Optimierung der Hyperparameter in dieser Arbeit nicht behandelt. Dies kann auf den jeweiligen Datensätzen zu besseren Ergebnissen führen und ist daher eine weitere Empfehlung zu weiterführenden Forschungen. Um das \textit{perceptual-aliasing} Problem zu behandeln, wurde in der Literatur raumzeitliche Informationen aus Bildsequenzen von aufeinanderfolgenden Frames berücksichtigt \cite{walchImageBasedLocalizationUsing2017, clarkVidLocDeepSpatioTemporal2017}. Des Weiteren wurde PoseNet an einem \textit{Bayessian Neural Network} angepasst, um die Unsicherheit der Ergebnisse zu modellieren. Dadurch ist es in der Evaluationsphase möglich, den Ergebnissen zu vertrauen oder diese zu verwerfen \cite{kendallModellingUncertaintyDeep2016}. Angesichts dessen können weitere Forschungsprojekte die hier erhobenen Datensätze auf die Nachfolger von PoseNet \cite{kendallModellingUncertaintyDeep2016, walchImageBasedLocalizationUsing2017, clarkVidLocDeepSpatioTemporal2017} anwenden und die Lokalisierungfähigkeit vergleichen.