% !TEX root = ../my_thesis.tex

\section{Fazit}
\label{sec:kapitel_6}
Zusammenfassend konnte in der vorliegenden Arbeit festgestellt werden, dass das KNN PostNet keinen adäquaten Ersatz zur Erhebung der realen Daten durch synthetische Daten ermöglichen konnte. PostNet konnte die Akkurattessen mit ... Fehlerrate und in einem begrenzten Areal in einer Richtung erheben.

Damit liegt die Schlussfolgerung nahe, dass PostNet die sehr zeit- und kostenaufwändige reale Datenerhebung nicht mit einer synthetischen Datenerhebung ersetzen kann.


\subsection{Ausblick}

%datentypen
\citet{acharyaBIMPoseNetIndoorCamera2019} erzielten ihre besten Ergebnisse durch das Trainieren mit den Gradientenbilder der synthetischen Kantenbilder (\textit{grad-edge}). In dieser Arbeit konnte eine bessere Akkuratesse mit den \textit{grad-cartoon} Datensätzen häufiger erzielt werden (s. Tab. \ref{tab:results_ic}- \ref{tab:results_hs_stairs_down}). Dennoch kann auf dieser Tatsache kein synthetischer Datentyp als der Beste festgelegt werden, da hierzu die bestmögliche Akkuratesse von jedem Datentyp weder sichergestellt noch ausgeschlossen werden konnte. Das wird wie in \ref{subsec:disc_methode} erwähnt dadurch begründet, dass die Akkuratesse vom Zufall abhängig ist. Daher können weiterführende Forschungen bei gleichen Hyperparametern die Anzahl der Trainingsprozesse erhöhen, um bessere Ergebnisse zu erzielen oder auszuschließen. Infolgedessen könnte der Zusammenhang zwischen den Ergebnissen und der Datentypen untersucht werden. 

Außerdem wurde die Optimierung der Hyperparameter in dieser Arbeit nicht behandelt. Dies kann auf den jeweiligen Datensätzen zu besseren Ergebnissen führen und ist daher eine weitere Empfehlung zu weiterführenden Forschungen. Um das \textit{perceptual-aliasing} Problem zu behandeln, wurde in der Literatur raumzeitliche Informationen aus Bildsequenzen von aufeinanderfolgenden Frames berücksichtigt \cite{walchImageBasedLocalizationUsing2017, clarkVidLocDeepSpatioTemporal2017}. Des Weiteren wurde PoseNet an einem Bayessian Neural Network angepasst, um die Unsicherheit der Ergebnisse zu modellieren. Dadurch ist es in der Evaluationsphase möglich, den Ergebnisse zu vertrauen oder zu verwerfen \cite{kendallModellingUncertaintyDeep2016}. Angesichts dessen können weitere Forschungsprojekte die hier erhobenen Datensätze auf die Nachfolger von PoseNet \cite{kendallModellingUncertaintyDeep2016, walchImageBasedLocalizationUsing2017, clarkVidLocDeepSpatioTemporal2017} anwenden und die Fähigkeit zur Lokalisierung untersuchen.