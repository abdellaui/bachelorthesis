% !TEX root = ../my_thesis.tex

\section{Fazit}
\label{sec:kapitel_6}
Das Zeil dieser Bachelorarbeit war es, den Ansatz zur Pose Estimation anhand von Convolutional Neuroal Networks und simulierten 3D Daten von \citet{acharyaBIMPoseNetIndoorCamera2019} in größeren Gebäudesimulationen und auf \textit{längeren} Strecken zu untersuchen.
Zusammenfassend konnte in der vorliegenden Arbeit festgestellt werden, dass ein Lokalisierungsverfahren über das Convolutional Neuroal Network PoseNet durch das Tranieren bei übereinstimmenden Hyperparametern wie in \cite{acharyaBIMPoseNetIndoorCamera2019} von den hier erhobenen synthetischen Daten nicht möglich ist. Eine durchschnittliche Akkuratesse ergab mit ... Fehlerrate und 

PoseNet konnte mit einer durchschnittlichen Akkuratesse von 10.95$m$ in der Position und 49.69° in der Orientierung auf einem begrenzten Gebäudeareal in einer Richtung trainiert werden.

Damit liegt die Schlussfolgerung nahe, dass PostNet die sehr zeit- und kostenaufwändige reale Datenerhebung nicht mit einer synthetischen Datenerhebung ersetzen kann.
