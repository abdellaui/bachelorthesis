% !TEX root = ../my_thesis.tex

\section{Fazit}
\label{sec:kapitel_6}
Das Ziel dieser Bachelorarbeit war, den Ansatz von \citet{acharyaBIMPoseNetIndoorCamera2019} zur Pose Estimation anhand von Convolutional Neural Networks und simulierten 3D-Daten in größeren Gebäudesimulationen und auf längeren Strecken zu untersuchen.
Insgesamt wurde der Ansatz in zwei unterschiedlichen Gebäuden auf vier Strecken untersucht. Zusammenfassend konnte durch diese Untersuchungen festgestellt werden, dass ein Lokalisierungsverfahren über den Ansatz von \citet{acharyaBIMPoseNetIndoorCamera2019} mit den hier erhobenen Datensätzen nicht möglich ist.


Eine Akkuratesse von ca. 1$m$ in der Position und 8° in der Orientierung wurde durch das zugrundeliegende CNN PoseNet mit den Daten der gleichen Domäne in den Trainings- und Evaluationsphasen erzielt. Im Gegensatz dazu wurde bei der domänenübergreifenden Anwendung eine durchschnittliche Akkuratesse von 10.95$m$, 49.69° erreicht. Hierbei konnte PoseNet nur auf einem begrenzten Gebäudeareal in einer Richtung trainiert werden. Dies zeigte Parallelen zu den Ergebnissen von \citet{acharyaBIMPoseNetIndoorCamera2019}, da diese PoseNet auf einer überwiegend in eine Richtung verlaufenden Strecke in einem kleinen Korridor trainierten. Angesichts dieser Parallelen lag die Schlussfolgerung nahe, dass durch das Trainieren mit den Gradientenbildern der synthetischen Daten bei gleichen Hyperparametern wie in \cite{acharyaBIMPoseNetIndoorCamera2019} von PoseNet die Gradientenbilder der realen Evaluationsdaten auf einem begrenzten Gebäudeareal nur in eine Richtung bestimmt werden können.

% Motivieren
Ein Lokalisierungsverfahren mit einer Positionsakkuratesse von ca. 11$m$ durch die domänenübergreifende Anwendung von PoseNet ist undenkbar. Selbst die potenzielle Akkuratesse von ca. 1$m$, 8° wäre im direkten Gebrauch für Anwendungen wie z.B. die automatische Bauforstschritterfassung sowie das Facility-Manage\-ment und die Navigation über Augmented Reality unzureichend. Allerdings könnte bei einer Positionsakkuratesse von ca. 1$m$, um die nötige Akkuratesse anzustreben, die Pose lokal auf dem Kaskadeneffekt beruhend beispielsweise mit weiterführenden Netzwerken oder modellbasierten Ansätzen korrigiert werden.
% OUTLOOK

Die unzureichende Akkuratesse der domänenübergreifenden Anwendung von PoseNet liegt den Simulationsdefiziten wie z.B. das Fehlen von diversen Objekten und die Existenz von domänenspezifischen Artefakten zugrunde (vgl. \cite{acharyaBIMPoseNetIndoorCamera2019}). Es wäre in diesem Zusammenhang lohnenswert, zu untersuchen, ob durch die Diskrepanzminimierung zwischen den synthetischen Datensätzen und den realen Daten wie z.B. über \textit{Generative Adversarial Networks} zu einem Fortschritt führt. Ferner könnte ein Fortschritt durch eine Beschränkung der möglichen Posen im Trainingsprozess erzielt werden, sodass die Positionsergebnisse immer in den betroffenen Innenräumen der Gebäuden liegen.
