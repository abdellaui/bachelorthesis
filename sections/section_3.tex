% !TEX root = ../my_thesis.tex

\section{Training des Convolutional Neural Networks}
cnns werden erst modelliert und anschließend traniert.
posenet caffe implementierung
Trainingsdaten sind die synthetischen Daten und Testdaten sind die realen Daten

\subsection{Erhebung der echten Daten}

Die Aufnahmen der echten Daten wurden mit einem Konstrukt, bestehend aus 

Für die Aufnahmen wurden zeitgleich zwei unterschiedliche Kameras der Intel Realsense Reihe verwenden. Es wurde eine Intel Realsense T265, die über Inertial Measurement Units (\textit{IMU}) und zwei Fischaugenkameras die Odometrie ermittelt, eingesetzt. Zudem wurde eine Intel Realsense D435, die eine 3D Punktwolke, ein Tiefelbilb sowie ein RGB-Bild einer Szene liefert, benutzt. Die T265 wurde über die D435 montiert, siehe Abbildung .... Über das Robot Operating System (\textit{ROS}) Framework wurden die Kameras zeitgleich angesprochen und der Datenfluss synchronisiert. Somit beinhaltet jeder Datensatz zwei Bildern der Fischaugenkameras, ein Tiefelbild, ein RGB-Bild, eine 3D Punktwolke und die dazugehörige Odometrie pro Szene. Für diese Arbeit sind nur die RGB-Bilder der D435 sowie die Odometrie-Daten der T265 relevant.

Zuerst wurden für die Aufgabenstellung interessante Aufnahmezonen im Gebäudesimulation festgelegt und anschließend diese über eine
Intel RealSense T265 D435, ROS; imu/image data


\subsection{Generierung der synthetischen Daten}
Blender,
eigenes Addons erstellt ein Kamerakonstrukt, welches ein NURBs-Pfad entlag aufnimmt; Variation der Orientierung; ...

\subsection{Verarbeitung der Daten}
// gradienten\\
Bei der Erzeugung von Gradienten- bzw. Kantenbilder gehen einerseits wichtige Informationen im Hinblick auf das Ursprungsbild verloren, andererseits bleiben wichtige Informationen wie z.B. die geometrische Struktur erhalten.

\subsection{Trainingsparameter}
optimizer, beta,
learningrate,
weights decay, anz. data; ...