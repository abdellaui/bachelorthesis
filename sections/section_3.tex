% !TEX root = ../my_thesis.tex

\section{Methodik}

Die vorliegende Arbeit versucht den Ansatz von \citet{acharyaBIMPoseNetIndoorCamera2019} auf längeren Strecken in größeren Gebäudesimulationen zu untersuchen, worin das PoseNet Modell mit Gradienten- bzw. Kantenbilder der synthetischen Daten trainiert und mit Gradientenbilder der realen Daten evaluiert wird.

Die Architektur der künstlichen neuronalen Netzen bzw. CNNs werden grundsätzlich zuerst aufgabenspezifisch konstruiert und anschließend die Netze mit entsprechenden Daten trainiert. In der vorliegenden Arbeit wird als Netzwerk das in Abschnitt \ref{sec:posenet} beschriebene PoseNet Modell verwendet. 


Die Arbeit folgt den Ansatz von \citet{acharyaBIMPoseNetIndoorCamera2019} mit Gradienten- bzw. Kantenbilder der synthetischen Daten das Netzwerk zu trainieren, um anschließend mit Gradientenbilder der realen Daten das Netzwerk zu evaluieren. 


Im weiteren Verlauf dieses Kapitel werden die Datensätze angegeben. Danach wird die Erhebung / Generierung der realen / synthetischen Daten beschrieben und die Verarbeitung der Daten dargestellt. Im Anschluss werden die Trainingsparameter angegeben. 
